\section{Conclusion}
Although this paper failed to obtain positive results, it nonetheless provides a foundation 
for future work in the area of Machine Learning applied to P2P lending. In this paper, a linear 
and two tree-based ensemble classifiers were applied to the problem and failed to get good results.
Though it may be possible to train a different model successfully, it is likely that the best 
aspect to explore in future work would be the feature representation. For example, textual analysis 
of the loan description may give insight into the writers likelihood of defaulting. Additionally, 
many of the numerical features, such as income and loan amount, may not scale linearly. If so, it may 
be useful to bin them at different thresholds, or perhaps inlcude as a feature the square root or log
of their values. Furthermore, this paper only considered loans that had run to completion. Though more 
difficult to model, there is potential to learn from loans that are currently in progress. Lastly, 
different outputs could be explored. Rather than outputting a probability of success, it could be 
possible to train a regressor to directly compute the expected rate of return. If successful, the 
output of the regressor could be used directly to choose notes without the potential translation 
error. Using the methods outlined in this paper as a baseline, any or all of these approaches 
can be taken and hopefully lead to a model that will give prospective investors an edge when 
selecting notes.
\label{conclusion}
