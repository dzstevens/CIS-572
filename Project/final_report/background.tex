\section{Background}
\label{background}
The goal of almost any investor is achieve maximum return-on-ivestment. This is usually done by balancing potential profits 
with known risks. As stated previously, the primary risk in P2P lending is the risk that a note will default before it has 
been fullly paid off. Lending Club encapsulates a note's risk in a score called a \emph{loan grade}, which ranges from A to G
(A being riskier than B, B riskier than C, etc.). This score is primarily based on the borrowers credit history. While credit 
history is surely a very large factor in the default risk, it is unlikely to be a perfect predictor. An ideal system would 
be one that perfectly divines whether or not a note will default. Then, all an investor would need to do to maximize returns 
would be to have the system output all available notes that won't default and then choose the ones that will offer the 
largest returns (based on the note's interest rate). Returns are maximized and risk is fully mitigated!

Of course, this oracle-like system is most likely not possible. However, it does give insight into the problem at hand. Although 
we cannot perfectly predict if a note will default, we \emph{can} build a classifier that tries to more accurately compute the 
default \emph{probability} of a given loan. These probabilities will give the investors a better understanding of the risk 
when selecting notes.\footnote{A more informative approach may be to train a \emph{regressor} that directly predicts return. 
However, initial attempts at this performed abysmally and were abandoned.}

Although there has been much research into using machine learning techniques to try to predict the stock market, 
a literature search has not returned any previous accademic work investigating how those techniques might be applied 
to P2P learning. There have been several blog posts\cite{superfacts, smartpeerlending} that talk about using support 
vector machine classifiers to predict Lending Club note default rates, with varied success. In this paper, 
random forest and gradient boosting ensemble clasifiers are trained on the problem, and it is the first documented 
study to do so, to the best of the author's knowledge.
