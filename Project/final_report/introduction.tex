\section{Introduction}
\label{introduction}
With a traditional investment, a prospective investor purchases some sort of property or asset that he or she
believe will aprreciate in value. Larger variance in the asset's expected value means larger potential returns,
but it also means greater risk. The investor must consider this carefully to determine how much risk he or she 
can tolerate.

Peer-to-peer (P2P) lending breaks the mold of traditional investments. Rather than purchasing assets, a prospective 
investor lends his or her capital to a set of borrowers. The borrow must then repay the lender with interest. 
In this model, the maximum return is known up front. The risk does not come from variance in value, but rather 
from the possibility that the loan could default before it has been fully paid-off.

Lending Club\cite{lendingclub} is a fairly new company that matches borrowers with investors. With Lending Club, someone seeking a 
loan can apply to be entered into the system. Lending club performs a credit history analysis on the applicant and 
then either denies their application if they don't meet strict requirements, or assigns them a score (A through G) 
that determines the interest rate of their loan and (theoretically) implies its risk level. Rates range from about 
7.5\% for A loans to around 22.6\% for G loans. 

Once a loan has been posted, ivestors have the possibility of funding it. Funding is done by buying \emph{notes}. A 
note is \$25-worth of funding in one loan. This system allows investers to mitigate the risk of loans defaulting by 
only investing small amounts in a large number of loans. Lending Club claims that everyone who has purchased 800+ 
notes (\$20,000 worth) has acheived positive returns, with 93.76\% of ivestors experiencing loans between 6\% and 18+\%.

Lending Club publishes all of their historical and current loan data, with extensive information about each loan. 
Though the above rates of return sound decent, a natural question to ask is whether this data can be used to achieve 
an `edge' when selecting notes for investment. The goal of this paper is to apply standard Machine Learning algorithms 
to this data to generate a classifier 
that can be used to choose the loans that will lead to a maximum return. In particular, the strategies used are logistic 
regression, random forest classification and gradient boosting classification. The former classifier is a simple model 
used as a baseline for evaluation. The latter two are more sophisticated ensemble methods that tend to have 
very good performance in practice.

The remainder of our paper is organized as follows.
Section~\ref{background} provides a more detailed descrption of the problem and highlights previous related work. 
Section~\ref{methods} details the approach that was taken to solve this problem, including key design decisions. 
The systems developed are analyzed in Section~\ref{experiments}, including performance on real data.
Finally, Section~\ref{conclusion} summarizes contributions and discusses future work.
